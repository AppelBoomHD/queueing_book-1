% arara: pdflatex: { shell: yes, interaction: nonstopmode }
% arara: pythontex: {verbose: yes, rerun: modified }
% arara: pdflatex: { shell: yes, interaction: nonstopmode }
% arara: clean: { extensions: [ aux, blg, idx, ilg, ind, log, out, pytxcode, rel, toc ] }
% !arara: clean: { files: [ ans.tex, hint.tex] }



% arara: pdflatex
% arara: clean: { extensions: [ aux, blg, idx, ilg, ind, log, out, pytxcode, rel, toc ] }
% !arara: clean: { files: [ ans.tex, hint.tex] }


\documentclass[queueing_book]{subfiles}
\externaldocument{queueing_book}

\opt{solutionfiles,check}{
\loadgeometry{tufte}
\Opensolutionfile{hint}
\Opensolutionfile{ans}
}

\begin{document}

\section{Little's Law}
\label{sec:littles-law}

There is an important relation between the average sojourn time of a job and the long-run time-average number of jobs contained in the system.
This relation is called \emph{Little's law}, and is one of the most useful results in queueing theory.\sidenote{In fact, to analyze throughput in any input-output system.}
The aim of this section is to prove this law under some simple conditions.


We start by defining a few intuitively useful concepts. From~\cref{eq:14} we see that
\begin{equation*}
\frac 1 t\int_0^t \L(s)\, \d s = \frac 1 t\int_0^t (A(s)-D(s)) \, \d s
\end{equation*}
is the time-average of the number of jobs in the system during
$[0,t]$.
Next,\sidenote{Check \cref{fig:atltdt} to see how $\J_k$ and $\L(t)$ relate.}
the sojourn time of the $k$th job is the time between the moment the job arrives and departs, that is,
\begin{equation*}
 \J_k = \int_0^\infty \1{A_k \leq s < D_k}\,\d s.
\end{equation*}

Consider a departure time $T$ at which the system is empty so that $A(T) = D(T)$.
Then, for $k\leq A(T)$, 
\begin{equation*}
 \J_k = \int_0^T \1{A_k \leq s < D_k}\,\d s,
\end{equation*}
and for $s\leq T$,
\begin{equation*}
\L(s) = \sum_{k=1}^\infty \1{A_k \leq s < D_k} = \sum_{k=1}^{A(T)}\1{A_k \leq s < D_k}.
\end{equation*}

Now realize that at time $T$ the areas $\int_0^T \L(s)\, \d s$ and $\sum_{k=1}^{A(T)} \J_k$ are the same,\sidenote{\cref{ex:59}} that is, 
\begin{equation}\label{eq:111}
 \int_0^T \L(s)\, \d s = \sum_{k=1}^{A(T)} \J_k.
\end{equation}
Assuming that there exists an infinite number of times $\{T_i\}$,  $T_i\to\infty$, such that the system is empty, i.e.,  $A(T_i) = D(T_i)$, it is then evident that
\begin{equation*}
 \frac 1 {T_i} \int_0^{T_i} \L(s)\, \d s = \frac{A(T_i)} {T_i} \frac{1}{A(T_i)} \sum_{k=1}^{A(T_i)} \J_k.
\end{equation*}
Taking limits\sidenote{Provided they exist.} then gives \recall{Little's law}: 
\begin{equation*}
 \E \L = \lambda \E \J.
\end{equation*}
Note that Little's law need not hold at all moments in time,\sidenote{\cref{ex:43} and \cref{ex:44}} it is a statement about \emph{averages}.


\newthought{With the PASTA property} and Little's law it becomes quite easy to derive simple expressions for $E\J$ and $\E\W$ for the $M/M/1$ queue.\sidenote{For the $M/G/1$ queue, things are bit subtler, see \cref{ex:l-216} and~\cref{sec:mg1}.}
With PASTA, it follows for the $M/M/1$ queue that
\begin{equation}\label{eq:wqes}
 \E{\J} = \E \L \E S.
\end{equation}
Combining this with $\E\J = \E \W + \E S$ leads right away to\sidenote{\cref{ex:l-215}}
 \begin{align}\label{eq:110}
 \E\J &= \frac{\E S}{1-\rho}, & \E \L &= \frac\rho{1-\rho}, & \E{\Q} = \frac{\rho^2}{1-\rho}.
 \end{align}


 \newthought{In the above proof} of Little's law we assumed that there is an increasing sequence of epochs $\{T_k, k=0,1,\ldots\}$ at which the system is empty.
 However, in many practical queueing situations the system is never empty.
 In~\cref{ex:l-221} and \cref{ex:l-222} we show that, for Little's law to holds, it suffices that the system is rate-stable.



\begin{exercise}\label{ex:42}
  Use\marginpar{With this check, you can prevent making an often-made mistake.}
  the (physical) dimensions of the components of Little's law to check that $\E{\J} \neq \lambda \E{\L}$.
\begin{solution}
  The dimension of $\lambda$ is a \emph{rate} and $\E \J$ is time, hence the dimension of $\lambda \E \J$ is a number, just like $\E \L$.
\end{solution}
\end{exercise}


\begin{exercise}\label{ex:37}
 Consider the server of the $G/G/1$ queue as a system by itself.
Jobs arrive at rate $\lambda$ and stay $\E S$ in this system.
Use  Little's law to conclude that  $\rho = \lambda  \E S$.
\begin{solution}
Recall that  $\rho := \lim_{t\to\infty} t^{-1}\int_0^t \Ls(s)\d s$ for the $G/G/1$ queue. 
\end{solution}
\end{exercise}


\begin{exercise}\label{ex:43}
Compute $\E{\W(M/M/1)}$ \marginpar{$\E{\W (M/M/1)}$ is the waiting time in queue for the $M/M/1$ queue} when $\lambda=5/h$ and $\mu=6/h$.
\begin{solution}
$\E\W = \E \J - \E S = 1/6/(1-5/6) - 1/6 = 5/6h$.
\end{solution}
\end{exercise}

\begin{exercise}\label{ex:44}
 Suppose that, at the moment you join the system of~\cref{ex:43}, the number of customers in the system is 10.
What is your  expected waiting time?
\begin{solution}
  When you arrive, you first have to wait for the job in service to finish and then for the 9 jobs in queue.
  By the memoryless property, at the moment you arrive, the remaining service time of the job in service is $\E S$ Thus, you have to wait $10/\mu = 5/3 h \neq 5/6 h$.
\end{solution}
\end{exercise}



\begin{exercise}\label{ex:59}
Derive \cref{eq:111}.
\begin{hint}
 Substitute the definition of $\L(s)$ in the LHS, then reverse the integral and summation.
\end{hint}
\begin{solution}
\begin{equation*}
 \begin{split}
 \int_0^T \L(s)\, \d s & = \int_0^T \sum_{k=1}^{A(T)} \1{A_k \leq s < D_k}\, \d s \\
& = \sum_{k=1}^{A(T)}\int_0^T \1{A_k \leq s < D_k}\, \d s = \sum_{k=1}^{A(T)} \J_k.
 \end{split}
\end{equation*}
\end{solution}
\end{exercise}


\begin{exercise}\label{ex:l-216}
Why is~\cref{eq:wqes} \emph{not} true in general for the $M/G/1$ queue? 
\begin{solution}
 By the memoryless property of the (exponential) distributed service times of the $M/M/1$ queue, the duration of a job in service, if any, is $\Exp(\mu)$ also at an arrival moment.
 Therefore, at an arrival moment, all jobs in the system (whether in service or not) have the same expected duration.
 Hence, the expected time to spend in queue is the expected number of jobs in the system times the expected service time of each job, i.e., $\E{\J} = \E \L \E S$.
 Note that we use PASTA to see that the expected number of jobs in the system at an arrival is $\E{\L}$.
 For the $M/G/1$ queue, the job in service (if any) does not have the same distribution as a job in queue.
 Hence, the expected time in queue is not $\E \L \E S$.
\end{solution}
\end{exercise}

\begin{exercise}\label{ex:l-215}
Derive~\cref{eq:110}.
\begin{solution}
\begin{align*}
 \E\J &= \E \L \E S + \E S = \lambda \E \J \E S + \E S= \rho \E \J + \E S, \\
 \E \L &= \lambda \E\J = \frac{\lambda \E S}{1-\rho} = \frac\rho{1-\rho}, \\
 \E{\W} &= \E\J - \E S = \frac{\E S}{1-\rho} - \E S = \frac{\rho}{1-\rho} \E S,\\
 \E{\Q} &= \lambda \E{\J} = \frac{\rho^2}{1-\rho}, \\
 \E{\Ls} &= \E \L - \E{\Q} = \frac{\rho}{1-\rho} - \frac{\rho^2}{1-\rho} = \rho, 
\end{align*}
\end{solution}
\end{exercise}



\begin{exercise}\label{ex:l-220}
Explain that $\sum_{k=1}^{A(t)} \J_k \geq \int_0^t \L(s) \d s \geq \sum_{k=1}^{D(t)} \J_k$.
\begin{solution}
 Intuitively, the left term is all the work that arrived up to time $t$, the middle term is all the work that has been processed, and the right term all the work that left.
 Any job that is half-way its service counts for full at the left, for half in the middle expression, and not in the right.

 More formally, for any job $k$ and time $t$, we have $\J_k \1{A_k \leq t} \geq \int_0^t \1{A_k \leq s < D_k } \d s \geq \J_k \1{D_k \leq t}$. (To see this, fix $k$, and check the three cases $t < A_k, A_k \leq t < D_k, D_k < t$.) Then,
 \begin{equation*}
 \sum_{k=1}^\infty \J_k \1{A_k \leq t} \geq \int_0^t \sum_{k=1}^\infty \1{A_k \leq t < D_k} \d s \geq \sum_{k=1}^\infty \J_k \1{D_k \leq t}. 
 \end{equation*}
 Finally, note that $ \sum_{k=1}^\infty \J_k \1{A_k \leq t} = \sum_{k=1}^{A(t)} \J_k$ and $ \sum_{k=1}^\infty \J_k \1{D_k \leq t} = \sum_{k=1}^{D(t)} \J_k$, and use the definition of $\L(s)$.
\end{solution}
\end{exercise}

\begin{exercise}\label{ex:l-221}
Take suitable limits\marginpar{Continuation of \cref{ex:l-220}} to show that 
$\lambda \E\J \geq \E \L \geq \delta \E\J$.
 Where do you need the strong law of large numbers?
\begin{solution}
 \begin{equation*}
 \lim_{t\to\infty} \frac{A(t)}{t}\frac 1{A(t)}\sum_{k=1}^{A(t)} \J_k \geq \lim_{t\to \infty} \frac 1 t \int_0^t \L(s) \d s \geq \lim_{t\to\infty} \frac{D(t)}{t}\frac 1{D(t)} \sum_{k=1}^{D(t)} \J_k. 
 \end{equation*}
 We use the strong law of large numbers to conclude that the limits converges to $n^{-1} \sum_{k=1}^n \J_k \to \E \J$, and we assume that $\{\J_k, k\geq N\}$ is a sequence of i.i.d.\ random variables for $N$ sufficiently large.
\end{solution}
\end{exercise}


\begin{exercise}\label{ex:l-222}
 Suppose that $A(t) = \lambda t$ and $D(t)= {[A(t) - 10]}^+$ so that the system is never empty for $t>0$. 
Conclude that   Little's law is still true. 
\begin{solution}
$A(t) = D(t) + 10$ for $t>10$.  However, since $\delta=\lambda$,  Little's law follows by~\cref{ex:l-221}.
\end{solution}
\end{exercise}


\opt{solutionfiles}{\Closesolutionfile{hint}
\Closesolutionfile{ans}
\loadgeometry{normal}
\input{hint}
\input{ans}
}

\end{document}


%%% Local Variables:
%%% mode: latex
%%% TeX-master: t
%%% End:
